%================ch1======================================
\chapter{Introduction}\label{ch:ch1}

\section{Gaia Mission}
We have come a long way in observing the skies, from the naked eye to the mural instrument, from telescopes to space missions. The Gaia mission \citep{gaia} \citep{gaiafaccs} is one such astrometry mission launched in 2013. The spacecraft aims to measure the positions, distances and motions of celestial objects. The mission aims to be the most precise 3D catalogue of the sky, mapping every object it can while also measuring their motions which can give clues about the origin and evolution of our galaxy, the Milky Way.

The mission aims to determine the position and parallax of around a billion stars to an accuracy of around 20 $\mathrm{\mu}$as. The catalogue is set to be released in stages. The first data release took place in 2016 \citep{gaiadr1}, based on 14 months of observation. It inlcuded the positions and magnitudes for over a billion stars using only Gaia data. The second data release took place in 2018 \citep{gaiadr2}, after 22 months of observations, and improved on the precision of the previous release, as well as adding parallaxes and proper motions.

\section{Star Clusters}
Gravity is one of the 4 fundamental forces, and 




